%!TEX program = pdflatex
\documentclass{article}

% Opening
\title{
	   Búsqueda Binaria: Magic Ship\\
	   \texttt{{\large (https://codeforces.com/problemset/problem/1117/C)}}}
\author{Carlos Luis Aguila Fajardo}
\date{}

% Packages
\usepackage{amsmath}
\usepackage{amsthm}
\usepackage{amssymb}
% \usepackage{relsize}
\usepackage[utf8]{inputenc}

% Teoremas, demostraciones, proposiciones, lemas y definiciones
\newtheoremstyle{default}
	{\topsep}
	{\topsep}
	{}
	{0pt}
	{\bfseries}
	{:}
	{4pt plus 1pt minus 1pt}
	{}
\theoremstyle{default}
\newtheorem{theorem}{Teorema}
\newtheorem*{theorem*}{Teorema}
\newtheorem*{lemma}{Lema}
\newtheorem*{note}{Nota}
\newtheorem*{proposition}{Proposición}
\newtheorem*{demonstration}{Demostración}
\newtheorem*{definition}{Definición}
\begin{document}
\maketitle

\section{Introducción:}
\subsection{Problema análogo:}
	Se puede observar que el problema subyacente a éste es el siguiente: Sean $(x_1, y_1), (x_2, y_2)$ dos posiciones o puntos de entrada, y defínanse
%
	\begin{align*}
		D &= \{ (-1,0), (1,0), (0,-1), (0,1) \}\\
		D^* &= D \cup \{(0,0)\}
	\end{align*}
%	
	el conjunto de transformaciones que se pueden aplicar a cualquier punto, o por simplicidad, que se pueden aplicar a $(x_1, y_1)$ para llegar a $(x_2, y_2)$. Entonces, sean:
%	
	\begin{equation*} 
		\{d_1, d_2 \dots d_m\},\; d_i \in D
	\end{equation*}
%	
	transformaciones predefinidas de entrada, encontrar, si existe, el mínimo entero $k$ tal que existe el conjunto $T_k = \{t_1, \dots t_k\}$ de parejas $t_i$ tales que:
%
	\begin{align}
		\label{T_1}
		t_i = (d_r, x_i),\quad i \equiv r(m), \; x_i \in D^*\\
		\label{T_2}
		(x_1,y_1) + \sum\limits_{i=1}^{k}{d(t_i) + x(t_i)} = (x_2, y_2)
	\end{align}
%
	Es decir, el conjunto $T_k$ de parejas de transformaciones tal que aplicando todas las transformaciones de $T_k$ sobre $(x_1,y_1)$ se obtiene $(x_2,y_2)$. Nótese que toda pareja de $t_i = (d_r, x_i)$ se compone de una transformación de entrada y una incógnita $x_i$, no es objetivo de este problema determinar las $x_i$, solo el mínimo tamaño $k$ para dicho conjunto.

	Problemas como este, con esta posibilidad limitada de movimiento por \\cuadrículas son usualmente referidos como problemas de geometría del taxista, o de distancia \textit{Manhattan}.
%
%
\newpage
%
%
\section{Solución general}
%
\subsection{Definiciones y proposiciones}
	\begin{definition}
		Llamamos distancia \textit{Manhattan} entre dos puntos $(x_1,y_1), (x_2,y_2)$ a la suma de las diferencias absolutas de sus coordenadas:
		\begin{equation*}
			dM((x_1,y_1),(x_2,y_2)) = |x_1 - x_2| + |y_1 - y_2|
		\end{equation*}
	\end{definition}
	%
	\begin{proposition}
		La distancia \textit{Manhattan} entre dos puntos $(x_1,y_1), (x_2,y_2)$ también define la cantidad mínima de transformaciones normales $d_i \in D$ que se necesitan para llegar a $(x_2,y_2)$ desde $(x_1,y_1)$.
	\end{proposition}
	%
	\begin{definition}
		Sea $T = \{t_1, t_2, \dots \}$ el conjunto infinito de parejas $t_i = (d_r, x_i)$ que cumplen (\ref{T_1}) y (\ref{T_2}), denotamos $T_k$ un prefijo tamaño $k$ de $T$.
	\end{definition}
	%
	\begin{definition}
		Un valor $k$ se dice válido, aunque no necesariamente mínimo, si es posible llegar de $(x_1, y_1)$ a $(x_2, y_2)$ con $T_k$.
	\end{definition}
	%
	\begin{definition}
		Un valor $k$ es solución del problema si es válido y mínimo.
	\end{definition}
%
%
\subsection{Teoremas}
	\begin{lemma}
		Sea $T_k$ aplicado sobre un punto $(x_1,y_1)$ para llegar a $(x_2,y_2)$, $k$ es válido si y solo si existe un punto $(x_p, y_p)$ tal que es posible llegar de $(x_1,y_1)$ a $(x_p, y_p)$ con $\{d_1 \dots d_k\}$ y de $(x_p, y_p)$ a $(x_2, y_2)$ con $\{x_1 \dots x_k\}$.
	\end{lemma}


	\begin{demonstration}
		Es fácil ver que si $k$ es válido entonces se cumple que:
		\\

		$(\implies)$\vspace*{-27pt}
		\begin{align*}
			(x_1,y_1) + \sum\limits_{i=1}^{k} { \Big( d(t_i) + x(t_i) \Big)} &= (x_2, y_2) \\
			(x_1,y_1) + \sum\limits_{i=1}^{k}{d(t_i)} + \sum\limits_{i=1}^{k}{x(t_i)} &= (x_2, y_2)
		\end{align*}
		sea $(x_p, y_p) = (x_1, y_1) + \sum\limits_{i=1}^{k}{d(t_i)}$, entonces:
		\begin{equation*}
			(x_p,y_p) + \sum\limits_{i=1}^{k}{x(t_i)} = (x_2, y_2)
		\end{equation*}

		$(\impliedby)$ Por otro lado, si existe $(x_p, y_p)$ se cumple que:
		\begin{align*}
			(x_1,y_1) + \sum\limits_{i=1}^{k}{d(t_i)}
				&= (x_p, y_p)\\
			(x_1,y_1) + \sum\limits_{i=1}^{k}{d(t_i)} + \sum\limits_{i=1}^{k}{x(t_i)}
				&= (x_p,y_p) + \sum\limits_{i=1}^{k}{x(t_i)}\\
			(x_1,y_1) + \sum\limits_{i=1}^{k}{\Big((t_i) + x(t_i)\Big)}
				&= (x_2,y_2)
		\end{align*}
		entonces $k$ es válido.
	\end{demonstration}
%
	\begin{theorem}\label{thm:valid-k}
		Sea $(x_p,y_p) = (x_1,y_1) + \sum\limits_{i=1}^k{d(t_i)}$, $k$ es válido si se cumple:
		\begin{equation*}
			dM((x_p,y_p), (x_2,y_2)) \leq k
		\end{equation*}
	\end{theorem}
	\begin{demonstration}
		Ampliando la ecuación anterior obtenemos:
		\begin{align*}
			dM((x_p,y_p), (x_2,y_2)) &\leq k \\
			|x_p - x_2| + |y_p - y_2| &\leq k \\
			|x_p - x_2| + |y_p - y_2| &= k - \alpha	,\quad 0 \leq \alpha \leq k
		\end{align*}
		de forma que existen $x_i$ tal que:
		\begin{align*}
			(x_p, y_p) + \sum\limits_{i=1}^{k-\alpha}{x(t_i)}
					   + \sum\limits_{i=1}^{\alpha}{(0,0)}
					   &= (x_2, y_2)
			\\
			(x_p, y_p) + \sum\limits_{i=1}^{k}{x(t_i)}
					   &= (x_2, y_2)
			\\
			(x_1,y_1) + \sum\limits_{i=1}^k{d(t_i)}
					  + \sum\limits_{i=1}^{k}{x(t_i)}
					   &= (x_2, y_2)
		\end{align*}
		entonces $k$ es válido.
	\end{demonstration}
%
	\paragraph{Nota:}Naturalmente a la hora de implementar los algoritmos, basta con tener en cuenta que $k$ es solución si es el primer entero válido.
%
	\begin{theorem}\label{thm:valid-k-max}
		Si existe una solución $k$ al problema, esta nunca supera el valor $m \cdot dM((x_1,y_1),(x_2,y_2))$
	\end{theorem}
	\begin{demonstration}
		Consideremos las transformaciones de entrada $d = \{d_1, \dots d_m\}$: dada su naturaleza cíclica (las transformaciones se repiten para un $T_k$ con $k > m$), en el peor de los casos existe una solución, pero por cada ciclo de $m$ transformaciones solo se reduce la distancia entre $(x_1,y_1), (x_2,y_2)$ en 1. De tal forma que la solución sería realizar tantos ciclos como la distancia entre estos puntos, y cada ciclo está compuesto por $m$ transformaciones. Por tanto la peor solución posible que pueda existir es $m \cdot dM((x_1,y_1),(x_2,y_2))$.
	\end{demonstration}
%
%
\section{Algoritmos}
	Notemos que para encontrar una solución basta con encontrar el menor entero no negativo $k$ que es válido. Conocemos además por el \textbf{Teorema \ref{thm:valid-k-max}} que la solución no supera nunca $m \cdot d$ donde $d$ es la distancia \textit{Manhattan} entre los puntos de entrada.
%
\subsection{Búsqueda entera \textit{naive}}
	Un acercamiento a solución utilizando las demostraciones anteriores sería utilizar fuerza bruta para obtener el $k$ mínimo posible. Iterando todo $k$ desde $0$ hasta $n \cdot d$ comprobamos si se cumple:
	\begin{equation*}
		dM((x_p,y_p), (x_2,y_2)) \leq k, \quad\; (x_p,y_p) = (x_1,y_1) + \sum\limits_{i=1}^{k}{d(t_i)}
	\end{equation*}

	\subsubsection{Complejidad temporal}
		En el peor caso, no existe un $k$ válido, por tanto el algoritmo iteraría todos los enteros del intervalo $[0,dM((x_1,y_1),(x_2,y_2))]$. Sea $n = max\{x_1, y_1, x_2, y_2\}$, el algoritmo realizaría $O(nm)$ iteraciones.

		Por cada iteración sin embargo, se calcula la suma de a lo sumo $m$ transformaciones, por lo que la complejidad temporal final es $O(nm^2)$.
%
%
\subsection{Búsqueda entera binaria}
	\subsubsection{Observaciones}
		Como consecuencia del \textbf{Teorema \ref{thm:valid-k}}, si existe un $k$ válido, todo $k^\prime > k$ es también válido. Por tanto si utilizamos un predicado $P(k)$:
		\begin{equation*}
			P(k) = dM((x_p,y_p), (x_2,y_2)) \leq k, \quad\; (x_p,y_p) = (x_1,y_1) + \sum\limits_{i=1}^{k}{d(t_i)}
		\end{equation*}
		la lista hipotética de enteros de $0$ a $n \cdot d$ es de la forma:
		\begin{verbatim}
		        [ False, False, ... False, True, True, ... True ]
		\end{verbatim}
		por lo cual se puede realizar una búsqueda binaria con predicado sobre la misma.
%
	\subsubsection{Mejoras en complejidad temporal}
		Considerando que ahora se recorren los enteros de forma logarítmica, y los enteros se pueden acotar a $O(nm)$, entonces la Complejidad temporal del algoritmo es $O(m\log(nm))$, donde la $m$ fuera del logaritmo es el costo del predicado.
%
%
\subsection{Mejoras utilizando sumas de prefijos}
	Consideremos el cálculo de la suma $\sum\limits_{i=1}^{k}{d(t_i)}$, donde siempre se cumple que $k = m\cdot\Big\lfloor\frac{k}{m}\Big\rfloor + r,\ k \equiv r(m)$. Entonces:
	\begin{equation*}
		\sum\limits_{i=1}^{k}{d(t_i)} 
			= \cdot \Bigg\lfloor\frac{k}{m}\Bigg\rfloor\sum\limits_{i=1}^{m}{d(t_i)}
			+ \sum\limits_{i=1}^{r}{d(t_i)}
	\end{equation*}
	de forma tal que cualquier suma de $d(t_i)$ se puede expresar en sumas de ciclos de $m$ transformaciones mas un resto, es decir, un ciclo no completado.

	Sea $S$ la lista de sumas de prefijos, precalculada en $O(m)$:
	\begin{equation*}
		S = \left[
		\sum\limits_{i=1}^{1}{d(t_i)},
		\sum\limits_{i=1}^{2}{d(t_i)},
		\dots
		\sum\limits_{i=1}^{m}{d(t_i)},
	\right]
	\end{equation*}
	entonces hallar $(x_p,y_p)$ en las búsquedas naive y binaria se convertiría en una operación $O(1)$, por lo que la complejidad temporal de los algoritmos se puede reducir a $O(nm)$ y $O(m + \log(nm))$ respectivamente.
%
%
\subsection{Implementación y comprobación de casos}
	En el archivo \texttt{scripts/magic\_ship.py} se encuentra la implementación de la búsqueda binaria mejorada.

	Una versión mas simple, el \textit{naive} con mejoras de sumas de prefijo, fue implementada en \texttt{scripts/tester.py} para comprobar los resultados de la búsqueda binaria sobre los casos prueba generados. Para ejecutar la generación de casos basta con ejecutar los comandos:
	\begin{verbatim}
	    cd scripts
	    python tester.py n
	\end{verbatim}
	donde $n$ representa la cantidad de pruebas aleatorias a generar.
%
%
%
\end{document}